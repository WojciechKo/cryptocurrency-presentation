%% bare_conf.tex
%% V1.4b
%% 2015/08/26
%% by Michael Shell
%% See:
%% http://www.michaelshell.org/
%% for current contact information.
%%
%% This is a skeleton file demonstrating the use of IEEEtran.cls
%% (requires IEEEtran.cls version 1.8b or later) with an IEEE
%% conference paper.
%%
%% Support sites:
%% http://www.michaelshell.org/tex/ieeetran/
%% http://www.ctan.org/pkg/ieeetran
%% and
%% http://www.ieee.org/

%%*************************************************************************
%% Legal Notice:
%% This code is offered as-is without any warranty either expressed or
%% implied; without even the implied warranty of MERCHANTABILITY or
%% FITNESS FOR A PARTICULAR PURPOSE!
%% User assumes all risk.
%% In no event shall the IEEE or any contributor to this code be liable for
%% any damages or losses, including, but not limited to, incidental,
%% consequential, or any other damages, resulting from the use or misuse
%% of any information contained here.
%%
%% All comments are the opinions of their respective authors and are not
%% necessarily endorsed by the IEEE.
%%
%% This work is distributed under the LaTeX Project Public License (LPPL)
%% ( http://www.latex-project.org/ ) version 1.3, and may be freely used,
%% distributed and modified. A copy of the LPPL, version 1.3, is included
%% in the base LaTeX documentation of all distributions of LaTeX released
%% 2003/12/01 or later.
%% Retain all contribution notices and credits.
%% ** Modified files should be clearly indicated as such, including  **
%% ** renaming them and changing author support contact information. **
%%*************************************************************************


% *** Authors should verify (and, if needed, correct) their LaTeX system  ***
% *** with the testflow diagnostic prior to trusting their LaTeX platform ***
% *** with production work. The IEEE's font choices and paper sizes can   ***
% *** trigger bugs that do not appear when using other class files.       ***                          ***
% The testflow support page is at:
% http://www.michaelshell.org/tex/testflow/




\documentclass[journal,12pt,onecolumn,draftclsnofoot,]{IEEEtran}
% Some Computer Society conferences also require the compsoc mode option,
% but others use the standard conference format.
%
% If IEEEtran.cls has not been installed into the LaTeX system files,
% manually specify the path to it like:
% \documentclass[conference]{../sty/IEEEtran}





% Some very useful LaTeX packages include:
% (uncomment the ones you want to load)


% *** MISC UTILITY PACKAGES ***
%
%\usepackage{ifpdf}
% Heiko Oberdiek's ifpdf.sty is very useful if you need conditional
% compilation based on whether the output is pdf or dvi.
% usage:
% \ifpdf
%   % pdf code
% \else
%   % dvi code
% \fi
% The latest version of ifpdf.sty can be obtained from:
% http://www.ctan.org/pkg/ifpdf
% Also, note that IEEEtran.cls V1.7 and later provides a builtin
% \ifCLASSINFOpdf conditional that works the same way.
% When switching from latex to pdflatex and vice-versa, the compiler may
% have to be run twice to clear warning/error messages.






% *** CITATION PACKAGES ***
%
%\usepackage{cite}
% cite.sty was written by Donald Arseneau
% V1.6 and later of IEEEtran pre-defines the format of the cite.sty package
% \cite{} output to follow that of the IEEE. Loading the cite package will
% result in citation numbers being automatically sorted and properly
% "compressed/ranged". e.g., [1], [9], [2], [7], [5], [6] without using
% cite.sty will become [1], [2], [5]--[7], [9] using cite.sty. cite.sty's
% \cite will automatically add leading space, if needed. Use cite.sty's
% noadjust option (cite.sty V3.8 and later) if you want to turn this off
% such as if a citation ever needs to be enclosed in parenthesis.
% cite.sty is already installed on most LaTeX systems. Be sure and use
% version 5.0 (2009-03-20) and later if using hyperref.sty.
% The latest version can be obtained at:
% http://www.ctan.org/pkg/cite
% The documentation is contained in the cite.sty file itself.






% *** GRAPHICS RELATED PACKAGES ***
%
\ifCLASSINFOpdf
  % \usepackage[pdftex]{graphicx}
  % declare the path(s) where your graphic files are
  % \graphicspath{{../pdf/}{../jpeg/}}
  % and their extensions so you won't have to specify these with
  % every instance of \includegraphics
  % \DeclareGraphicsExtensions{.pdf,.jpeg,.png}
\else
  % or other class option (dvipsone, dvipdf, if not using dvips). graphicx
  % will default to the driver specified in the system graphics.cfg if no
  % driver is specified.
  % \usepackage[dvips]{graphicx}
  % declare the path(s) where your graphic files are
  % \graphicspath{{../eps/}}
  % and their extensions so you won't have to specify these with
  % every instance of \includegraphics
  % \DeclareGraphicsExtensions{.eps}
\fi
% graphicx was written by David Carlisle and Sebastian Rahtz. It is
% required if you want graphics, photos, etc. graphicx.sty is already
% installed on most LaTeX systems. The latest version and documentation
% can be obtained at: 
% http://www.ctan.org/pkg/graphicx
% Another good source of documentation is "Using Imported Graphics in
% LaTeX2e" by Keith Reckdahl which can be found at:
% http://www.ctan.org/pkg/epslatex
%
% latex, and pdflatex in dvi mode, support graphics in encapsulated
% postscript (.eps) format. pdflatex in pdf mode supports graphics
% in .pdf, .jpeg, .png and .mps (metapost) formats. Users should ensure
% that all non-photo figures use a vector format (.eps, .pdf, .mps) and
% not a bitmapped formats (.jpeg, .png). The IEEE frowns on bitmapped formats
% which can result in "jaggedy"/blurry rendering of lines and letters as
% well as large increases in file sizes.
%
% You can find documentation about the pdfTeX application at:
% http://www.tug.org/applications/pdftex




\usepackage{svg}


% *** MATH PACKAGES ***
%
\usepackage{amsmath}
% A popular package from the American Mathematical Society that provides
% many useful and powerful commands for dealing with mathematics.
%
% Note that the amsmath package sets \interdisplaylinepenalty to 10000
% thus preventing page breaks from occurring within multiline equations. Use:
%\interdisplaylinepenalty=2500
% after loading amsmath to restore such page breaks as IEEEtran.cls normally
% does. amsmath.sty is already installed on most LaTeX systems. The latest
% version and documentation can be obtained at:
% http://www.ctan.org/pkg/amsmath





% *** SPECIALIZED LIST PACKAGES ***
%
%\usepackage{algorithmic}
% algorithmic.sty was written by Peter Williams and Rogerio Brito.
% This package provides an algorithmic environment fo describing algorithms.
% You can use the algorithmic environment in-text or within a figure
% environment to provide for a floating algorithm. Do NOT use the algorithm
% floating environment provided by algorithm.sty (by the same authors) or
% algorithm2e.sty (by Christophe Fiorio) as the IEEE does not use dedicated
% algorithm float types and packages that provide these will not provide
% correct IEEE style captions. The latest version and documentation of
% algorithmic.sty can be obtained at:
% http://www.ctan.org/pkg/algorithms
% Also of interest may be the (relatively newer and more customizable)
% algorithmicx.sty package by Szasz Janos:
% http://www.ctan.org/pkg/algorithmicx




% *** ALIGNMENT PACKAGES ***
%
%\usepackage{array}
% Frank Mittelbach's and David Carlisle's array.sty patches and improves
% the standard LaTeX2e array and tabular environments to provide better
% appearance and additional user controls. As the default LaTeX2e table
% generation code is lacking to the point of almost being broken with
% respect to the quality of the end results, all users are strongly
% advised to use an enhanced (at the very least that provided by array.sty)
% set of table tools. array.sty is already installed on most systems. The
% latest version and documentation can be obtained at:
% http://www.ctan.org/pkg/array


% IEEEtran contains the IEEEeqnarray family of commands that can be used to
% generate multiline equations as well as matrices, tables, etc., of high
% quality.




% *** SUBFIGURE PACKAGES ***
%\ifCLASSOPTIONcompsoc
%  \usepackage[caption=false,font=normalsize,labelfont=sf,textfont=sf]{subfig}
%\else
%  \usepackage[caption=false,font=footnotesize]{subfig}
%\fi
% subfig.sty, written by Steven Douglas Cochran, is the modern replacement
% for subfigure.sty, the latter of which is no longer maintained and is
% incompatible with some LaTeX packages including fixltx2e. However,
% subfig.sty requires and automatically loads Axel Sommerfeldt's caption.sty
% which will override IEEEtran.cls' handling of captions and this will result
% in non-IEEE style figure/table captions. To prevent this problem, be sure
% and invoke subfig.sty's "caption=false" package option (available since
% subfig.sty version 1.3, 2005/06/28) as this is will preserve IEEEtran.cls
% handling of captions.
% Note that the Computer Society format requires a larger sans serif font
% than the serif footnote size font used in traditional IEEE formatting
% and thus the need to invoke different subfig.sty package options depending
% on whether compsoc mode has been enabled.
%
% The latest version and documentation of subfig.sty can be obtained at:
% http://www.ctan.org/pkg/subfig




% *** FLOAT PACKAGES ***
%
%\usepackage{fixltx2e}
% fixltx2e, the successor to the earlier fix2col.sty, was written by
% Frank Mittelbach and David Carlisle. This package corrects a few problems
% in the LaTeX2e kernel, the most notable of which is that in current
% LaTeX2e releases, the ordering of single and double column floats is not
% guaranteed to be preserved. Thus, an unpatched LaTeX2e can allow a
% single column figure to be placed prior to an earlier double column
% figure.
% Be aware that LaTeX2e kernels dated 2015 and later have fixltx2e.sty's
% corrections already built into the system in which case a warning will
% be issued if an attempt is made to load fixltx2e.sty as it is no longer
% needed.
% The latest version and documentation can be found at:
% http://www.ctan.org/pkg/fixltx2e


%\usepackage{stfloats}
% stfloats.sty was written by Sigitas Tolusis. This package gives LaTeX2e
% the ability to do double column floats at the bottom of the page as well
% as the top. (e.g., "\begin{figure*}[!b]" is not normally possible in
% LaTeX2e). It also provides a command:
%\fnbelowfloat
% to enable the placement of footnotes below bottom floats (the standard
% LaTeX2e kernel puts them above bottom floats). This is an invasive package
% which rewrites many portions of the LaTeX2e float routines. It may not work
% with other packages that modify the LaTeX2e float routines. The latest
% version and documentation can be obtained at:
% http://www.ctan.org/pkg/stfloats
% Do not use the stfloats baselinefloat ability as the IEEE does not allow
% \baselineskip to stretch. Authors submitting work to the IEEE should note
% that the IEEE rarely uses double column equations and that authors should try
% to avoid such use. Do not be tempted to use the cuted.sty or midfloat.sty
% packages (also by Sigitas Tolusis) as the IEEE does not format its papers in
% such ways.
% Do not attempt to use stfloats with fixltx2e as they are incompatible.
% Instead, use Morten Hogholm'a dblfloatfix which combines the features
% of both fixltx2e and stfloats:
%
% \usepackage{dblfloatfix}
% The latest version can be found at:
% http://www.ctan.org/pkg/dblfloatfix




% *** PDF, URL AND HYPERLINK PACKAGES ***
%
%\usepackage{url}
% url.sty was written by Donald Arseneau. It provides better support for
% handling and breaking URLs. url.sty is already installed on most LaTeX
% systems. The latest version and documentation can be obtained at:
% http://www.ctan.org/pkg/url
% Basically, \url{my_url_here}.




% *** Do not adjust lengths that control margins, column widths, etc. ***
% *** Do not use packages that alter fonts (such as pslatex).         ***
% There should be no need to do such things with IEEEtran.cls V1.6 and later.
% (Unless specifically asked to do so by the journal or conference you plan
% to submit to, of course. )


% correct bad hyphenation here
\hyphenation{op-tical net-works semi-conduc-tor}

\usepackage[utf8]{inputenc}
\usepackage[T1]{fontenc}

\def\abstractname{Abstrakt}
\def\figurename{Ryc.}

\def\refname{Bibliografia}

\begin{document}
%
% paper title
% Titles are generally capitalized except for words such as a, an, and, as,
% at, but, by, for, in, nor, of, on, or, the, to and up, which are usually
% not capitalized unless they are the first or last word of the title.
% Linebreaks \\ can be used within to get better formatting as desired.
% Do not put math or special symbols in the title.
\title{Architektura Kryptowalut\\Studium Bitcoina i Ethereum}


% author names and affiliations
% use a multiple column layout for up to three different
% affiliations
\author{
  \IEEEauthorblockN{Wojciech Korzeniowski}
  \IEEEauthorblockA{
     Instytut Informatyki\\
     Wydział Elektroniki i Technik Informacyjnych\\
     Politechnika Warszawska
  }
}

% conference papers do not typically use \thanks and this command
% is locked out in conference mode. If really needed, such as for
% the acknowledgment of grants, issue a \IEEEoverridecommandlockouts
% after \documentclass

% for over three affiliations, or if they all won't fit within the width
% of the page, use this alternative format:
% 
%\author{\IEEEauthorblockN{Michael Shell\IEEEauthorrefmark{1},
%Homer Simpson\IEEEauthorrefmark{2},
%James Kirk\IEEEauthorrefmark{3}, 
%Montgomery Scott\IEEEauthorrefmark{3} and
%Eldon Tyrell\IEEEauthorrefmark{4}}
%\IEEEauthorblockA{\IEEEauthorrefmark{1}School of Electrical and Computer Engineering\\
%Georgia Institute of Technology,
%Atlanta, Georgia 30332--0250\\ Email: see http://www.michaelshell.org/contact.html}
%\IEEEauthorblockA{\IEEEauthorrefmark{2}Twentieth Century Fox, Springfield, USA\\
%Email: homer@thesimpsons.com}
%\IEEEauthorblockA{\IEEEauthorrefmark{3}Starfleet Academy, San Francisco, California 96678-2391\\
%Telephone: (800) 555--1212, Fax: (888) 555--1212}
%\IEEEauthorblockA{\IEEEauthorrefmark{4}Tyrell Inc., 123 Replicant Street, Los Angeles, California 90210--4321}}




% use for special paper notices
%\IEEEspecialpapernotice{(Invited Paper)}




% make the title area
\maketitle

% As a general rule, do not put math, special symbols or citations
% in the abstract
\begin{abstract}
Opis koncepcji i mechanizmów wykorzystanych przy tworzeniu kryptowalut.
\end{abstract}

% no keywords




% For peer review papers, you can put extra information on the cover
% page as needed:
% \ifCLASSOPTIONpeerreview
% \begin{center} \bfseries EDICS Category: 3-BBND \end{center}
% \fi
%
% For peerreview papers, this IEEEtran command inserts a page break and
% creates the second title. It will be ignored for other modes.
\IEEEpeerreviewmaketitle

\section{Wstęp}

Niniejszy artykuł przedstawia jakie problemy z pieniędzmi istnieją w dzisiejszym świecie i w jaki sposób pojawienie się
kryptowalut może pomóc je rozwiązać. Następnie opiszę matematyczne koncepty które zostały wykorzystane przy
projektowaniu kryptowalut oraz samą architekturę kryptowalut. Dalej wyjaśnię czym jest Blockchain, co oznacza kopanie
bloków oraz jaki sposób zapewniane jest bezpieczeństwo. Następnie wyjaśnię czym jest fork w kontekście Blockchainu oraz
opiszę co nowego wprowadza Ethereum w stosunku do Bitcoina.

\subsection{Historia waluty}

Od zarania dziejów na świecie istniał handel. Zanim jednak postały waluty, ludzie wymieniali się między sobą rozmaitymi
dobrami w sposób bezpośredni.  Jeden z problemów który występuje podczas takiej wymiany jest problem z wydaniem reszty.
Przykładowo, jeżeli ktoś kto hodował świnie i potrzebował kostkę masła, musiał wymienić całą świnię na dużą ilość masła.
Ewentualnie mógł podzielić świnie i zostać z resztą świni co mogło powodować problem z jej przechowaniem. Rozwiązaniem
tego problemu okazały się waluty. Hodowca mógł sprzedać swoją świnie w zamian za określoną ilość danej waluty a
następnie część przeznaczyć na zakup masła. Inny problem który istnieje w świecie bez walut to problem z oszacowaniem
wartości jednych dóbr w stosunku do innych. Zdecydowanie łatwiej jest sprowadzić wartość dóbr do wspólnej waluty co
pozwala na łatwiejsze określenie w jakim stosunku powinna zostać dokonana wymiana jednego dobra z inne.

\subsection{Rola banków}

Jeżeli posiadamy pieniądze, ale nie chcemy być odpowiedzialni za ich przechowywanie możemy skorzystać z usług Banku.
Podstawową operacją jaką możemy wykonać w Banku jest możliwość zdeponowania oraz pobrania wcześniej zdeponowanych
środków. Kolejną z operacji jest wykonanie transferu środków z jednego konta na inne. Taka operacja umożliwia transfer
środków innej osobie bez konieczności bezpośredniego przekazania pieniędzy w postaci monet czy banknotów. W takiej
sytuacji Bank poświadcza że właściciel konta przekazał swoje środki innej osobie dzięki czemu ta może z nich skorzystać.
Taka rola sprawa że Banki są jedną z instytucji zaufania publicznego. Oznacza to to między innymi fakt że społeczeństwo
wierzy, iż w każdej chwili może odebrać powierzone Bankowi pieniądze a to jak pokazuje historia nie zawsze jest prawdą.
Za przykład posłuży sytuacja Grecji z 2015 roku gdzie w wyniku kryzysu finansowego greckie Banki zostały zamknięte a
wypłaty z bankomatów ograniczone do 60 euro na dzień.\cite{kryzysGrecji}

Fakt iż Bank jest odpowiedzialny za weryfikację czy dana osoba posiada odpowiednią ilość środków do wykonania transferu
wymusza prowadzenie rejestru. W rejestrze znajduje się historia wszystkim transakcji oraz stan konta przypisany dla
każdego użytkownika. Na banku spoczywa odpowiedzialność aby zawartość rejestru nie wpadła w niepowołane ręce oraz aby
rejestr nie przepał co spowodowałoby utratę zgromadzonych przez klientów środków gdyż tylko on jest dokumentem
poświadczającym stan konta.

\section{Matematyka}

W tym rozdziale omówię matematyczne koncepty które zostały wykorzystane przy projektowaniu kryptowalut, będę się do nich
odnosił w  dalszej części artykułu.

\subsection{Funkcja skrótu}

Funkcja skrótu jest to funkcja która przyporządkowuje dowolnemu ciągowi znaków, inny ciąg znaków o stałej długości.
Jedną z właściwości funkcji skrótu jest fakt iż po niewielkiej zmianie źródłowego ciągu znaków, wynik funkcji zmienia się
całkowicie, co widać na poniższych przykładach:

\begin{gather}
\text{SHA256} ('Alice') = 3bc5106297 \ldots a0699a3043 \\
\text{SHA256} ('Bob') = cd9fb1e148 \ldots bb4bb4e961 \\
\text{SHA256} ('Bob.') = ec46deb8be \ldots d035fd84a2
\end{gather}

Kolejną właściwością funkcji skrótu jest niemożliwość znalezienia źródłowego ciągu znaków posiadając tylko jego skrót.
Fakt ten sprawia że funkcję skrótu można wykorzystać do sprawdzania czy dwie wartości są takie same bez potrzeby przechowywania
oryginalnej wartości. Jest to wykorzystywane podczas logowania się użytkowników w serwisach internetowych. Dzięki temu
w bazie danych nie są przechowywane hasła w sposób jawny, zamiast tego przechowuje się jedynie ich skrót który jest
porównywany ze skrótem wprowadzonego hasła podczas logowania.

\subsection{Drzewo skrótów}

\begin{figure}[!t]
  \centering
  \includegraphics[width=3in]{img/merkle-tree}
  \caption{Drzewo skrótów}
  \label{fig:merkle}
\end{figure}

Jest to bardziej rozbudowana wersja funkcji skrótu dzięki której możemy otrzymać skrót z listy obiektów. W kontekście
kryptowalut wykorzystuje się drzewo skrótów do otrzymania skrótu grupy transakcji. Najpierw wyliczany jest skrót dla
pojedynczej transakcji później skróty są parowane a następnie z pary obliczany jest kolejny skrót. Operacja jest
powtarzana aż otrzymamy jeden skrót który reprezentuje skrót całej listy. Wizualizacja obliczania przedstawiona jest na
diagramie numer~\ref{fig:merkle} gdzie pojedyncza transakcja jest reprezentowana przez symbol \textit{Tx} a skrót
całości oznaczony jest jako \textit{Root Hash}.

\subsection{Podpis cyfrowy}

Podpis cyfrowy jest techniką która pozwala na weryfikację autora wiadomości. Osoba chcąca skorzystać z podpisu cyfrowego
musi posiadać klucz prywatny oraz klucz publiczny. Klucz prywatny służy do podpisywania wiadomości i powinien być znany
tylko i wyłącznie autorowi wiadomości. Z kolei klucz publiczny służy do weryfikacji czy wiadomość została podpisana
przez odpowiadający mu klucz prywatny. W wyniku podpisu wiadomości powstaje Sygnatura która zostaje przesłana wraz z
wiadomością. Odbiorca wiadomości wykorzystując Sygnaturę, klucz publiczny nadawcy oraz samą wiadomość jest w stanie
sprawdzić czy wiadomość została podpisana przy użyciu właściwego klucza prywatnego.

\section{Kryptowaluty}

Kryptowaluta jest to wirtualna waluta która nie ma swojej fizycznej reprezentacji. Jednak nie powinno się jej postrzegać
jako byt który ma mniejszą wartość niż waluty tradycyjne. Często można usłyszeć opinię iż kryptowaluty nie mają żadnej
wartości ponieważ nie są fizyczne i istnieją tylko wirtualne. Każda waluta sama w sobie nie ma żadnej wartości, nabiera
ją dopiero wtedy, kiedy możemy w zamian za nią otrzymać coś innego. Brak posiadania fizycznej postaci rozwiązuje problem
przechowywania dużych ilości pieniędzy bez konieczności korzystania z Banków. Z drugiej strony powoduje powstanie nowych
problemów z jej przechowywaniem oraz wykorzystaniem jako środek płatności jednak jest to kwestia świadomości jakie
istnieją zagrożenia i w jaki sposób można im zapobiec.

\subsection{Portfel}

Portfel stanowi para klucz prywatny oraz klucz publiczny. Klucz prywatny służy do podpisywania transakcji w których
pieniądze są przelewane z powiązanego portfela na inny portfel wskazany poprzez przypisany do niego klucz publiczny.
Klucz publiczny stanowi swego rodzaju adres na który można przesłać pieniądze. Natomiast klucz prywatny jest hasłem do
naszego konta bankowego bez którego nie jesteśmy w stanie wykonać przelewu.

Najczęściej klucze reprezentowane są jako ciąg znaków zakodowany przy pomocy base58. Dzięki zastosowaniu base58 klucze
są zapisywane w postaci ciągu liter i cyfr z pomięciem znaków które jest łatwo ze sobą pomylić, takich jak 0 (cyfra
zero) oraz O (litera O). Istnieje jednak możliwość zapisana kluczy w dowolny sposób który umożliwia odtworzenie
oryginalnego ciągu znaków. Do tego celu może zostać wykorzystany np.\ kod QR, obraz cyfrowy w którym na najmniej
znaczących bitach zapisana jest informacja o kluczach (steganografia), kilkanaście losowo wygenerowanych podczas
tworzenia portfela słów które można zapamiętać, zapisać lub dalej zakodować. Możliwości na przechowywanie klucza
prywatnego są ograniczone ludzką pomysłowością na zapis informacji w sposób zrozumiały tylko dla autora.

\subsection{Transakcja}

W celu wykonania transakcji użytkownik musi posiadać klucz prywatny. Transakcja zawiera w sobie informację na jaki
portfel powinna zostać przeniesiona określona ilość waluty. Po utworzeniu Transakcji musi ona zostać podpisana przy
pomocy klucza prywatnego powiązanego z portfelem z którego zostaje wykonany transfer. Następnie transakcja wraz z
podpisem zostaje wysłana do sieci Blockchain która zajmuje się weryfikacją transakcji oraz zapewnia prawidłowe działanie
całego systemu.

Istotny jest fakt iż utworzenie transakcji z podpisem oraz wysłanie jej do sieci nie muszą nastąpić w tym samym
momencie. W celu uzyskania najwyższego bezpieczeństwa zalecane jest wykonanie podpisu transakcji na komputerze bez
podłączenia do internetu. Następnie skopiowanie transakcji wraz z podpisem na komputer połączony z internetem i wysłanie
jej do sieci. W ten sposób zyskujemy większe bezpieczeństwo poprzez fakt iż klucz prywatny nie jest przechowywany ani
wprowadzany na komputerze z połączeniem internetowym który jest bardziej narażony na złośliwe oprogramowanie niż
odizolowana jednostka.

\section{Blockchain}

Blockchain jest sercem kryptowalut i stanowi rewolucję w dziedzinie rozproszonych baz danych. Blockchain składa się z
bloków który z kolei zawiera listę transakcji. Wielkość bloku jest ustalona i w przypadku Bitcoina wynosi nie więcej niż
1MB, co daje około 2000 transakcji na blok.\cite{transakcje} Bitcoin jest zaprojektowany tak aby nowy blok z
transakcjami pojawiał się co 10 minut co daje ostatecznie 3-4 transakcje na sekundę. Dla porównania PayPal realizuje
średnio 193 trasakcje na sekundę a Visa 1'667.\cite{porownanieTransakcji}

\subsection{Blok}

\begin{figure}[!t]
  \centering
  \includegraphics[width=3in]{img/blockchain}
  \caption{Schemat Blockchain}
  \label{fig:blockchain}
\end{figure}

Na Rycinie~\ref{fig:blockchain} przedstawiony został schemat Blockchainu oraz pojedynczego bloku. Każdy z bloków składa
się z listy transakcji oraz jej wartości jej skrótu wyliczonej przy użyciu drzewa skrótów, skrótu z poprzedniego bloku,
oraz wartości \textit{Nonce}. Wszystkie z wymienionych elementów są użyte do obliczenia skrótu danego bloku.

Każdy z bloków w Blockchainie musi spełnić warunek aby jego hash zaczynał się op określonej liczby zer. Aby to osiągnąć
do bloku została dodana wartość \textit{Nonce}. Zmiana jej wartości całkowicie zmienia wartość hash z bloku.

Podział na bloki i zawieranie wartości funkcji skrótu z poprzedniego bloku przypomina strukturę która została użyta do
budowy systemu Git\cite{git}. Całość tworzy liniową strukturę w której dowolna zmiana bloku z przeszłości (taka jak
dodanie lub usunięcie transakcji) powoduje zmianę hasha tego bloku co skutkuje przerwaniem łańcucha Blockchain ponieważ
kolejny blok zawiera hash zmienionego bloku sprzed modyfikacji. Aby zachować integralność z blokami występującymi dalej
w Blockchainie należy dla każdego z tych bloków ustawić nową wartość hash poprzedniego bloku.

\subsection{Kopanie bloków - Proof of work}

Jako że nie ma możliwości aby uzyskać źródło funkcji skrótu na podstawie jej wyniku jedyną możliwością na spełnienie
wymagania aby hash bloku zaczynał się od określonej liczby zer jest sprawdzanie kolejnych wartości \textit{Nonce} aż
trafimy na taką która spełnia to wymaganie. Znalezienie tej wartości jest bardzo czasochłonne i tym zadaniem zajmują się
kopacze bloków którzy w zamian za udostępnienie swojej mocy obliczeniowej dostają wynagrodzenie w danej kryptowalucie.
To dzięki temu w sieci pojawia się co raz więcej Bitcoinów. Pierwsze bloki w Blockchainie nie zawierały żadnych
transakcji a jedynie wynagrodzenie dla kopacza. Nagroda ta jest zmniejszana o połowę co 4 lat aż do osiągnięcia limitu
21 milionów Bitcoinów w sieci, wtedy kopacze przestaną dostawać wynagrodzenie za samo znajdowanie odpowiedniej liczby
\textit{Nonce}. Po osiągnięciu limitu kopacze będą dostawać wynagrodzenie w postaci prowizji za transakcje.  Prowizje
ustala autor transakcji, im większa prowizja, tym większa szansa na to że jego transakcja znajdzie się w kolejnym bloku.
W przypadku ustalenia zbyt małej prowizji istnieje ryzyko że transakcja nigdy nie zostanie zaakceptowania ponieważ
kopacze wybiorą transakcje z większą prowizją do następnego bloku. W czasie pisania artykułu w sieci znajdowało się
150'000 transakcji oczekujących na akceptację\cite{niezatwierdzoneTransakcje}.

Wraz ze wzrostem sieci liczba wymaganych zer może ulec zmianie. W sieci Bitcoin ustalenie liczby wymaganych zer
następuje raz na 2 tygodnie i dobierane jest tak aby wydobycie nowego bloku zajmowało średnio 10 minut.

\subsection{Atak 51\%}

W przypadku gdy ktoś chciałby skompromitować sieć Bitcoin poprzez zmianę jednej z historycznych transakcji, musi wykopać
jeszcze raz dany blok oraz wszystkie bloki które po nim występują. Dla przypomnienia, trudność wydobycia bloku jest
dostosowana tak aby całej sieci średnio zajmowało to 10 minut. Prawdopodobieństwo wydobycia dwóch lub więcej bloków
przez jednego kopacza szybciej niż reszta sieci jest bliska zeru. Warunkiem wymaganym do skutecznego ataku jest
posiadanie minimum 51\% mocy obliczeniowej sieci. W przeciwnym wypadku reszta sieci będzie w stanie szybciej wydobywać
nowe bloki co skutkuje powstaniem dłuższego łańcucha.

W przypadku istnienia wielu łańcuchów brany jest pod uwagę ten który jest dłuższy. Wynika to z teorii gier która zakłada
że gracz zyska więcej na przestrzeganiu zasad (akceptacja dłuższego łańcucha i próba wydobycia nowego bloku na jego
szczycie) niż na łamaniu zasad (próba kompromitacji łańcucha poprzez nadpisanie istniejącej historii lub wydobywanie
bloku na krótszym łańcuchu).

Przez tą niepewność odnośnie końcowych bloków łańcucha które mogą ulec zmianie poprzez zastąpienie go innymi blokami
istnieje określenie pewność bloku. Im blok jest dalej od końca łańcucha tym bardziej pewny się staje i maleje szansa na
to że zostanie podmieniony. Przyjmuje się że 6.\ blok od końca jest wystarczająco pewny i jest niewielka szansa na to że
w przyszłości ulegnie podmianie.

\section{Forki}

Istnieje możliwość celowego rozdzielenia łańcucha Blockchain. Najczęstszym z powodów jest zmiana architektury
kryptowaluty dzięki której poprawi się jej stabilność. Zazwyczaj po forku właściciele portfeli z bazową kryptowalutą
stają się właścicielami takiej samej ilości nowej waluty ile posiadali bazowej przed forkiem.

\subsection{Bitcoin Cash}

Jest pierwszym \textit{hard forkiem} Bitcoina. Oznacza to iż jego wersja nie jest kompatybilna wstecz z Bitcoinem.
Główną zmianą było zwiększenie wielkości bloku z 1MB do 8MB co według założeń twórców spowoduje iż Bitcoin Cash będzie
bardziej użyteczny przy transferach małych kwot. Sytuacja miała miejsce w momencie gdy za transakcję Bitcoina trzeba
było zapłacić kilkadziesiąt dolarów.

\subsection{Bitcoin Gold}

Kolejny fork Bitcoina który powstał 2.5 miesiąca po forku Bitcoin Cash. Przez wielu uznany za oszustwo i nie powinien
być traktowany na poważnie. Wprowadzone zmiany obejmują zmianę algorytmu wykorzystanego do wyliczania skrótu, zmiana z
SHA256 na Equihash. Dodatkowo dostosowanie trudności wydobycia bloku następuje po wykopaniu każdego bloku, w oryginalnym
Bitcoinie zmiana następuje co 2016 blok czyli około 2 tygodnie.

\subsection{SegWit2x}

Fork który nie doszedł do skutku jednak miał wpływ na wahania ceny Bitcoina. Zmiana obejmuje głównie podwojenie bloku w
stosunku do Bitcoina z 1MB na 2MB. Został odwołany ponieważ społeczność uznała iż zmiana niewiele wnosi biorąc pod uwagę
iż już istnieje Bitcoin Cash ze zwiększoną wielkością bloku i kolejny fork nie wniesie żadnej korzyści.

\section{Ethereum}

W przeciwieństwie do forków, Ethereum jest całkowicie nową walutą która wprowadza możliwość wprowadzenia własnego kodu
źródłowego do Blockchainu. Dzięki temu Ethereum jest w stanie zaoferować o wiele więcej niż transfery środków.

\subsection{Kontrakty}

Ethereum umożliwia utworzenie kontraktów. Dzięki temu mechanizmowi możliwe jest wykorzystanie blockchainu jako
rozproszonej bazy danych w której istnieje gwarancja że dane nie zostana zmienione ani usunięte. Kontrakt składa się z
własnej pamięcie oraz z funkcji które mogą korzystać z tej pamięci i wysyłać walutę na inny adres. Funkcje można
porównać do procedur znanych z relacyjnych baz danych przy pomocy których można się czytać z pamięcie kontraktu oraz ją
modyfikować. Czytanie pamięci kontraktu odbywa się bezpłatnie, natomiast jeżeli chcemy wprowadzić zmiany w pamięci
kontraktu zmuszeni jesteśmy do uiszczenia opłaty za tę operację. Kosz jest proporcjonalny do złożoności obliczeniowej
operacji którą chcemy wykonać.

Kontrakt który został umieszczony na Blockchainie nie może zostać już zmieniony. Może zostać utworzony nowy, zmieniony
kontrakt, jednak istniejącego nie można zmienić.

Najprostszym przykładem kontraktu jest dystrybucja Tokenów. Każdy może utworzyć swoje własne Tokeny na Blockchainie
Ethereum a następnie je sprzedawać za Ethereum. Przykładowo politycy mogą wydawać swoje własne Tokeny które mogą zostać
zakupione przez wyborców Dzięki temu politycy zyskują fundusze na działalność a kupujący wierzy w to że polityk jest
uczciwy i działa dla dobra społeczeństwa co powinno spowodować wzrost wartości jego Tokenów. Mechanizm został nazwany
ICO (Initial Coin Offering) i aktualnie staje się nową formą crowdfoundingu.

Innym kontraktem może być loteria. Uczestnicy wpłacają pieniądze na adres Kontraktu. Następnie dokonuje się losowanie w
wyniku którego wybrany zostaje jednej z uczestników który wygrywa całą pulę.

\section*{Podsumowanie}

Kryptowaluty zdają się być rewolucją w bankowości i podejściu do waluty na miarę rewolucji dokonanej przez internet w
dziedzinie komunikacji. Główną z zalet Kryptowalut jest oderwanie ich od fizycznej postaci, możliwość transferu do
dowolnego miejsca na Ziemi bez pośredników, przejrzystość oraz brak możliwości dodrukowania. Z góry wiadomo ile
maksymalnie może ich powstać oraz jakie są zasady dystrybucji. Według wizji wielu osób Banki, ubezpieczalnie, zakłady
bukmacherskie i wszelkie działalności w których występuje transfer pieniędzy będzie można zastąpić odpowiednią
Kryptowalutą, Konraktem czy kolejną ideą zbudowaną na na Blockchainie.

Pewne jest to że sama technologia Blockchain zrewolucjonizuje wiele sektorów ponieważ ma wiele do zaoferowania jako
szeroko pojęta rozproszona baza danych niekoniecznie związana z pieniędzmi. Już można znaleźć informacje o planowanych
migracjach systemów bankowych na Blockchain, wykorzystanie go do przechowywania danych medycznych, czy nawet
przeprowadzania wyborów.

% An example of a floating figure using the graphicx package.
% Note that \label must occur AFTER (or within) \caption.
% For figures, \caption should occur after the \includegraphics.
% Note that IEEEtran v1.7 and later has special internal code that
% is designed to preserve the operation of \label within \caption
% even when the captionsoff option is in effect. However, because
% of issues like this, it may be the safest practice to put all your
% \label just after \caption rather than within \caption{}.
%
% Reminder: the "draftcls" or "draftclsnofoot", not "draft", class
% option should be used if it is desired that the figures are to be
% displayed while in draft mode.
%
%\begin{figure}[!t]
%\centering
%\includegraphics[width=2.5in]{myfigure}
% where an .eps filename suffix will be assumed under latex, 
% and a .pdf suffix will be assumed for pdflatex; or what has been declared
% via \DeclareGraphicsExtensions.
%\caption{Simulation results for the network.}
%\label{fig_sim}
%\end{figure}

% Note that the IEEE typically puts floats only at the top, even when this
% results in a large percentage of a column being occupied by floats.


% An example of a double column floating figure using two subfigures.
% (The subfig.sty package must be loaded for this to work.)
% The subfigure \label commands are set within each subfloat command,
% and the \label for the overall figure must come after \caption.
% \hfil is used as a separator to get equal spacing.
% Watch out that the combined width of all the subfigures on a 
% line do not exceed the text width or a line break will occur.
%
%\begin{figure*}[!t]
%\centering
%\subfloat[Case I]{\includegraphics[width=2.5in]{box}%
%\label{fig_first_case}}
%\hfil
%\subfloat[Case II]{\includegraphics[width=2.5in]{box}%
%\label{fig_second_case}}
%\caption{Simulation results for the network.}
%\label{fig_sim}
%\end{figure*}
%
% Note that often IEEE papers with subfigures do not employ subfigure
% captions (using the optional argument to \subfloat[]), but instead will
% reference/describe all of them (a), (b), etc., within the main caption.
% Be aware that for subfig.sty to generate the (a), (b), etc., subfigure
% labels, the optional argument to \subfloat must be present. If a
% subcaption is not desired, just leave its contents blank,
% e.g., \subfloat[].


% An example of a floating table. Note that, for IEEE style tables, the
% \caption command should come BEFORE the table and, given that table
% captions serve much like titles, are usually capitalized except for words
% such as a, an, and, as, at, but, by, for, in, nor, of, on, or, the, to
% and up, which are usually not capitalized unless they are the first or
% last word of the caption. Table text will default to \footnotesize as
% the IEEE normally uses this smaller font for tables.
% The \label must come after \caption as always.
%
%\begin{table}[!t]
%% increase table row spacing, adjust to taste
%\renewcommand{\arraystretch}{1.3}
% if using array.sty, it might be a good idea to tweak the value of
% \extrarowheight as needed to properly center the text within the cells
%\caption{An Example of a Table}
%\label{table_example}
%\centering
%% Some packages, such as MDW tools, offer better commands for making tables
%% than the plain LaTeX2e tabular which is used here.
%\begin{tabular}{|c||c|}
%\hline
%One & Two\\
%\hline
%Three & Four\\
%\hline
%\end{tabular}
%\end{table}


% Note that the IEEE does not put floats in the very first column
% - or typically anywhere on the first page for that matter. Also,
% in-text middle ("here") positioning is typically not used, but it
% is allowed and encouraged for Computer Society conferences (but
% not Computer Society journals). Most IEEE journals/conferences use
% top floats exclusively. 
% Note that, LaTeX2e, unlike IEEE journals/conferences, places
% footnotes above bottom floats. This can be corrected via the
% \fnbelowfloat command of the stfloats package.



% trigger a \newpage just before the given reference
% number - used to balance the columns on the last page
% adjust value as needed - may need to be readjusted if
% the document is modified later
%\IEEEtriggeratref{8}
% The "triggered" command can be changed if desired:
%\IEEEtriggercmd{\enlargethispage{-5in}}

% references section

% can use a bibliography generated by BibTeX as a .bbl file
% BibTeX documentation can be easily obtained at:
% http://mirror.ctan.org/biblio/bibtex/contrib/doc/
% The IEEEtran BibTeX style support page is at:
% http://www.michaelshell.org/tex/ieeetran/bibtex/
\bibliographystyle{IEEEtran}
\bibliography{bibliography}

% that's all folks
\end{document}
